\documentclass{report}

\usepackage{amsmath}
\usepackage{amssymb}
\usepackage{amstext}

\usepackage{comment}

\newcommand{\paren}[1]{\left(#1\right)}
\newcommand{\biggparen}[1]{\bigg(#1\bigg)}

\newcommand{\bra}[1]{\left<#1\right|}
\newcommand{\ket}[1]{\left|#1\right>}
\newcommand{\braket}[2]{\left<#1|#2\right>}
\newcommand{\ketbra}[2]{\left|#2\right>\!\!\left<#1\right|}

\newcommand{\tr}{\text{\textbf{tr}}\,}

\newcommand{\Z}{\mathbb{Z}}

\begin{document}

\part*{Introduction}

It can be very useful to study systems in the infinite limit for two reasons.  First, because for sufficiently large systems (and finite-length boundary effects) the vast majority of the system will be in the bulk far from the boundaries and so the system will effectively act as if were infinitely large.  Second, because even if a system has non-trivial boundary effects, it is still useful to be able to break down the total behavior into (roughly speaking) a bulk properties component and a boundary effects component in order to better understand what is going on.

When working with infinite systems one needs to deal with the fact that all extensive physical quantities are divergent.  Fortunately, this is not a big deal because we are usually interested in the intensive quantities anyway.  To compute such quantities, we need to introduce some structure into our system.  In this report we focus on translationally invariant systems living on an infinite lattice, and in particular we assume that the cell consists of a single lattice site;  this last assumption is not strictly necessary because the methods we shall describe can be generalized to larger cell sizes, but to make things simpler we shall assume a cell size of one.

In order to simulate infinite systems, we need a representation that both adequately captures the physics we are interested in and also is amenable to numeric computation.  In this report we focus on tensor network states, which have proven to work well for this purpose [citations here].  The basic idea is that we decompose our system into a network of repeated tensors; each of these tensors has one rank that corresponds to the physical observable of a particle at that site, and two (in 1D) or four (in 2D) additional ranks that connect each site to its neighbors, as illustrated in [];  this representation effectively models entanglement in the system as a sort of local communication between lattice sites.

Having settled on using tensor network states as an ansatz, we need to have a way to extract information about physical quantities from the ansatz and a way to fit this ansatz so that it is a good approximation of the true state being modeled.  In both cases, we start by constructing an effective environment by contracting all of the tensors surrounding a particular lattice site; this gives us both a way to compute the expectations of local operators and a starting guess for the site which we improve the effective iteratively by absorbing improved sites into it.

In this report, we present our work in using these methods as the basis of simulation algorithms.  In the first part by reviewing the 1D variant of this approach in order to provide a background as well as to provide more concrete details than supplied in [X].  In the second part we present our efforts towards extending these concepts to 2D.  We end with a conclusion.

\part{1D Simulation Algorithm}
\label{1dsim}

In this part we review the 1D algorithm presented in \cite{Crosswhite2008}.  We do this rather than skipping to the 2D generalization because both 1D and 2D share characteristics in common and 1D is simpler so it is worth taking the time to thoroughly review it before generaalizing it to 2D.

Conceptually, we assume that we are working with a system whose ground state can be well-approximated by an infinitely long (in both directions) chain of site tensors $A$, as illustrated in [].  Informally, the mathematical representation of this wave function is given by
\begin{equation}
\label{bi-inf-1d-state-a}
\psi(\dots,\alpha_{-1},\alpha_{0},\alpha_1,\dots)= \cdots A^{\alpha_{-1}} \cdot A^{\alpha_0}\cdot A^{\alpha_1}\cdot A^{\alpha_2} \cdots,
\end{equation}
where the left-hand side is the wave function evaluated with a value for each observable in the system (which has an infinite number of of observables), and the right-hand side has an infinite product of matrices, $\prod_{i=-\infty}^{+\infty} A^{\alpha_i},$ where $A^{\alpha}$ is the matrix obtained by taking the rank-3 tensor $A$ and slicing it for a given value of the physical dimension, i.e. $(A^\alpha)_{ij}\equiv A^\alpha_{ij}$.

For the sake of brevity, we rewrite \ref{bi-inf-1d-state-a} as
\begin{equation}
\label{bi-inf-1d-state-b}
\psi(\{\alpha_i\}_{i\in\Z}) = \prod_{i\in\Z} A^{\alpha_i},
\end{equation}
where $\Z$ is the set of integers of $\{\alpha_i\}_{i\in\Z}$ is a bi-infinite sequence of $\alpha_i$'s.

In general the right-hand-side of \ref{bi-inf-1d-state-b} will not converge, but it will for special values of $A$ and $\{\alpha_i\}_{i\in\Z}$.  For example, if $A^\alpha_{ij}=M_{ij}$ where $M$ is an idempotent matrix, then $\psi(\{\alpha_i\}_{i\in\Z})=\prod_{i\in\Z} A^i = \prod_{i\in\Z} M = M$.  Unfortunately, this result takes the form of a matrix when what we would really like is a complex number representing the amplitude of the wave in this configuration.  To get a complex number, we need to apply boundary conditions.  We have a couple of choices.  First, we can apply periodic boundary conditions, which gives us $\psi(\{\alpha_i\}_{i\in\Z})=\tr M$.  Second, we can apply open boundary conditions $L$ and $R$ so that $\psi(\{\alpha_i\}_{i\in\Z})=L\cdot M\cdot R$  In this case, our result strongly depends on our choice of $L$ and $R$.

The preceding discussion shows that it is possible for there to be enough structure in the system that the product converges, but its requirement that $A^\alpha$ be completely independent of $\alpha$ is completely unrealistic, but without this structure we need to arrange for every choice of $\{\alpha_i\}_{i\in\Z}$ result in a converging product, which is non-trivial.

Fortunately, in practice we don't actually need or even want the full wave function, only its expectation value with respect to various operators.  For example, let $Z$ be the Pauli Z spin matrix;  then the expectation with respect to $Z$ at site 0 is equal to
$$\sum_{\{\alpha_k\in\{\uparrow,\downarrow\}\}_{k\in\Z}\atop\alpha'_0\in\{\uparrow,\downarrow\}}\psi\paren{\{\alpha_k\}_{k\in\Z}}\psi^*\paren{\{\alpha_i\}_{i\in\Z/\{0\}};\alpha'_0} Z_{\alpha_0\alpha'_0}=\tr(M\cdot Z),$$ where $$M_{\alpha_0\alpha'_0}=\sum_{\{\alpha_k\in\{\uparrow,\downarrow\}\}_{k\in\Z}}\psi\paren{\{\alpha_k\}_{k\in\Z}}\cdot\psi^*\paren{\{\alpha_i\}_{i\in\Z/\{0\}};\alpha'_0}.$$
You can think of the above equations as being like a zipper that traces all the observables except for the one at site zero, forming $M$.

This notion of rewriting into a dot product between a matrix that encodes the \emph{environment} of a given site and a matrix that represents the expectation of a 1-site operator at that site is a very powerful one because it gives us a way to work with infinite systems.  We now formalize this insight as follows.

\part{2D Simulation Algorithm}
\label{2dsim}

\part*{Bibliography}

\bibliography{report.bib}
\bibliographystyle{plain}

\end{document}