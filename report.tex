\documentclass{article}

\begin{document}

\part*{Introduction}

It can be very useful to study systems in the infinite limit for two reasons.  First, because for sufficiently large systems (and finite-length boundary effects) the vast majority of the system will be in the bulk far from the boundaries and so the system will effectively act as if were infinitely large.  Second, because even if a system has non-trivial boundary effects, it is still useful to be able to break down the total behavior into (roughly speaking) a bulk properties component and a boundary effects component in order to better understand what is going on.

When working with infinite systems one needs to deal with the fact that all extensive physical quantities are divergent.  Fortunately, this is not a big deal because we are usually interested in the intensive quantities anyway.  To compute such quantities, we need to introduce some structure into our system.  In this report we focus on translationally invariant systems living on an infinite lattice, and in particular we assume that the cell consists of a single lattice site;  this last assumption is not strictly necessary because the methods we shall describe can be generalized to larger cell sizes, but to make things simpler we shall assume a cell size of one.

In order to simulate infinite systems, we need a representation that both adequately captures the physics we are interested in and also is amenable to numeric computation.  In this report we focus on tensor network states, which have proven to work well for this purpose [citations here].  The basic idea is that we decompose our system into a network of repeated tensors; each of these tensors has one rank that corresponds to the physical observable of a particle at that site, and two (in 1D) or four (in 2D) additional ranks that connect each site to its neighbors, as illustrated in [];  this representation effectively models entanglement in the system as a sort of local communication between lattice sites.

Having settled on using tensor network states as an ansatz, we need to have a way to extract information about physical quantities from the ansatz and a way to fit this ansatz so that it is a good approximation of the true state being modeled.  In both cases, we start by constructing an effective environment by contracting all of the tensors surrounding a particular lattice site; this gives us both a way to compute the expectations of local operators and a starting guess for the site which we improve the effective iteratively by absorbing improved sites into it.

In this report, we present our work in using these methods as the basis of simulation algorithms.  In the first part by reviewing the 1D variant of this approach in order to provide a background as well as to provide more concrete details than supplied in [X].  In the second part we present our efforts towards extending these concepts to 2D.  We end with a conclusion.

\part{1D Simulation Algorithm}
\label{1dsim}

\part{2D Simulation Algorithm}
\label{2dsim}

\part*{Bibliography}

\bibliography{report.bib}
\bibliographystyle{plain}

\end{document}